\section{Hamiltonien}


\begin{align}
    \frac{\partial H}{\partial q_i}
    &= \frac{\partial}{\partial q_i} \Big[ \sum_{i=1}^N \frac{p_i^2}{2}
    + (1 - \cos(q_i - q_{i-1})) \Big] \\
%
    &= \frac{\partial}{\partial q_i}
    \Big[ \sum_{i=1}^N (1 - \cos(q_i - q_{i-1})) \Big] \\
%
    &= - \frac{\partial}{\partial q_i} (\cos(q_i - q_{i-1}) + \cos(q_{i+1} - q_i)) \\
%
    &= -  (-\sin(q_i - q_{i-1}) - \sin(q_{i+1} - q_i)(-1)) \\
%
    &= \sin(q_i - q_{i-1}) - \sin(q_{i+1} - q_i)
\end{align}

Pour le premier rotor ($i = 1$) on a
\[\begin{dcases}
    \frac{\partial H}{\partial q_1} = \sin(q_1) - \sin(q_2 - q_1)
    & \text{si collé à la paroi} \\
    \frac{\partial H}{\partial q_1} = -\sin(q_2 - q_1)
    & \text{sinon}
\end{dcases}\]

Pour le dernier rotor ($i = N$) on a
\[\begin{dcases}
    \frac{\partial H}{\partial q_N} = \sin(q_N - q_{N-1}) + \sin(q_N)
    & \text{si collé à la paroi} \\
    \frac{\partial H}{\partial q_N} = \sin(q_N - q_{N-1})
    & \text{sinon}
\end{dcases}\]



\section{Générateur infinitésimal}

$$P_t f(x) = \mathbb{E}_x [f(q_1(t), ..., q_N(t); p_1(t), ..., p_N(t))]$$



Soit $\phi_t(x) = (q(t), p(t))$. Alors,
$$P_t f(x) = f(\phi_t(x)) ???$$

$$Lf(x) = \lim_{t \rightarrow 0} \frac{P_t f(x) - f(x)}{t}$$

Donc,
\begin{align}
Lf(x) &= \frac{\partial}{\partial t} f(\phi_t(x))
= \sum \frac{\partial f}{\partial q_i} \frac{\partial q_i}{\partial t}
+ \sum \frac{\partial f}{\partial p_i} \frac{\partial p_i}{\partial t} \\
%
&= \sum \frac{\partial f}{\partial q_i} \dot q_i
+ \sum \frac{\partial f}{\partial p_i} \dot p_i \\
%
&= \nabla_q f \cdot \dot q + \nabla_p f \cdot \dot p \\
&= \nabla_q f \cdot \nabla_p H - \nabla_p f \cdot \nabla_q H \\
%&= \nabla_q f \cdot \nabla_p H - \nabla_p f \cdot \nabla V \\
&= \nabla_p H \cdot \nabla_q f - \nabla V \cdot \nabla_p f \\
& \implique L = \nabla_p H \cdot \nabla_q - \nabla V \cdot \nabla_p
%
%&= p \cdot \nabla_q f + \nabla_p f \cdot \dot p \\
%&= p \cdot \nabla_q f + \nabla_p f \cdot \nabla_p H
\end{align}

\newpage
Donc.............

1) Hamiltonien $H$ (qui peut inclure des termes stochastiques)

2) $\phi_t(x) = (q(t), p(t))$ solution de $H$ à partir de $x$

3) $f(\phi_t(x)) = f(q(t), p(t))$ une fonction d'etat qu'on veut calculer

4) $Lf = \frac{\partial}{\partial t} f
   = \nabla_p H \cdot \nabla_q f - \nabla_q H \cdot \nabla_p f$

5) $L = \nabla_p H \cdot \nabla_q - \nabla_q H \cdot \nabla_p$

6) $\frac{\partial}{\partial t} f = Lf \implique f = e^{tL} f_0$

\vspace{3.0mm}
Et puis...

1) Si $f$ est une fonction dont on veut calculer l'integral (l'esperance)
par rapport à une mesure (inconnue)

2) Et $\phi_t$ est un processus stochastique lié à cette mesure - comment ?

3) Donc on peut juste calculer $\frac{1}{N} \sum f(\phi_t)$ ???

\vspace{3.0mm}
En fait !!!!!!!!!!!!!!!!!!!!!!

Le generateur est juste la dérivé de l'esperance de la formule d'ito ?????

$$
\mathrm{d}(f(X_t,t)) = \frac{\partial f}{\partial t}(X_t,t)\mathrm{d}t
+ \frac{\partial f}{\partial x}(X_t,t)\mathrm{d}X_t
+ \frac{1}{2}\frac{\partial^2 f}{\partial x^2}(X_t,t)\sigma_t^2\mathrm{d}t
$$

$$
    dX_t= \mu_t\,\mathrm{d}t + \sigma_t\,\mathrm{d}B_t
$$

\begin{align}
    dq_t &= -\nabla V(q_t) dt + \sqrt{\frac{2}{\beta}} dW_t \\
    \implique df &= \frac{\partial f}{\partial t}dt
    + \frac{\partial f}{\partial x}
    \left( -\nabla V(q_t) dt + \sqrt{\frac{2}{\beta}} dW_t \right)
    + \ud \frac{\partial^2 f}{\partial x^2}   \frac{2}{\beta} dt \\
%
    &= \frac{\partial f}{\partial t}dt
    + \frac{\partial f}{\partial x}
    \left( -\nabla V(q_t) dt + \sqrt{\frac{2}{\beta}} dW_t \right)
    + \frac{1}{\beta} \frac{\partial^2 f}{\partial x^2} dt \\
\end{align}
