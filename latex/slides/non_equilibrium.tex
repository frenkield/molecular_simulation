\begin{frame}

    \frametitle{Partie I \\
        Synthèse - Dynamique hors équilibre}

    Synthèse des sections 7.2, 7.3, 7.4, et 7.5 des notes de cours :

    \textbf{Introduction à la physique statistique numérique}
    \cite{stoltz_phys_stat}

    Gabriel Stoltz.

    \small \url{http://cermics.enpc.fr/~stoltz/Cours/intro_phys_stat.pdf}

\end{frame}

\begin{frame}

    \frametitle{Synthèse - Dynamique hors équilibre}

    La dynamique hors équilibre caractérise un système qui n'est pas
    réversible. Physiquement, un tel système présente un flux d'énergie
    (chaleur) d'une partie du système vers une autre.

    Ici, on suscite une dynamique hors équilibre, et donc un flux
    d'énergie, en appliquant au système des forçages thermiques et
    mécaniques.

    Lorsque le gradient de température est faible (et sans forçage),
    le flux reste linéaire, et la loi de Fourier s'en applique. En
    revanche, dès que le gradient de température est plus important,
    la dynamique devient beaucoup plus compliquée. Il n'existe pas
    une théorie générale dans ce cas.

\end{frame}

\begin{frame}

    \frametitle{Synthèse - Dynamique hors équilibre}

    En générale il n'est pas possible de déterminer analytiquement
    la mesure invariante d'un système hors équilibre. En particulier,
    cette mesure dépend des détails de la dynamique d'une manière
    non-triviale à cause des corrélations de longue portée.

    Par exemple, pour la dynamique perturbée,
    %
    \[dq_t = (-V'(q) + F)dt + \sqrt{2} dW_t,\]
    %
    la mesure invariante $\psi_F$ est
    %
    \[\psi_F(q) = Z_F^{-1} \int_0^1 e^{V(q+y) - V(q) -Fy} dy.\]
    %
    Si $F \neq 0$, cette mesure dépend des valeurs de $V$ partout.

\end{frame}



%$$\psi_F(q) = Z_F^{-1} \int_0^1 e^{V(q+y) - V(q) -Fy} dy
%= Z_F^{-1} e^{-V(q)} \int_0^1 e^{V(q+y)} e^{-Fy} dy.$$


%Pour la mesure invariante quand F \neq 0: si on change V qq part,
%l'expression de la mesure invariante (une sorte de convolution)
%montre que la densité change partout, de manière non triviale ;
%
%En contraste avec le cas d'une mesure de Gibbs dont la densité
%est proportionnelle à e^{-\beta V(q)} dq, et donc si on change
%localement la valeur de V, on ne change la densité de la mesure
%que localement (à un facteur de rescaling près).
%
%Le mieux pour vous en convaincre serait de faire une quadrature
%numérique pour calculer \psi_F, de changer V localement
%(pex ajouter un créneau à support petit), et regarder comment
%ça change la mesure globale.





















% calcul flux energie - stoltz page 33
% loi fourier

%    TODO
%    \begin{frame}
%        \frametitle{Calcul coefficients de transport}
%        Flux d'énergie...
%        Température !!!!!!!!!!!!!!!!!!
%    \end{frame}












% A very interesting question is whether nonequilibrium systems
% are locally close to equilibrium.
% c'est quoi ca ?
% file:///Users/frenkield/Library/Mobile%20Documents/com~apple~CloudDocs/cours/cours-spe%CC%81cialise%CC%81s/5MM50-mole%CC%81culaire/slides/rotors-alessandra-Iacobucci.pdf



% Donc... Fokker-Planck est juste l'EDP qui satisfait la mesure.


% sections 7.2 à 7.4 [et j'aurais discuté qq éléments de la section 7.5]


%\begin{frame}
%
%    \frametitle{Dynamique hors équilibre}
%
%    La dynamique hors équilibre caractérise un système qui n'est pas
%
%    \begin{equation}
%        \begin{dcases}
%            dq_{t} = M^{-1} p_{t} d t \\
%            dp_{t} = \left(-\nabla V\left(q_{t}\right) + \eta F\right)
%            dt-\gamma M^{-1} p_{t} d t + \sqrt{\frac{2 \gamma}{\beta}} dW_{t}
%        \end{dcases}
%        \label{eq:dynamique}
%    \end{equation}
%
%    % It is precisely because the perturbation is not of gradient type that
%    % some particle flux can appear in the steady-state.
%
%    % Attaching the chain on one side is important to remove the translation invariance of the whole system.
%
%    % η, and the proportionality constant is called the mobility.
%
%
%    We believe that there exists a unique smooth station-ary measure for the
%    dynamics (2). However, as far as weknow, there is no rigorous result in
%    this direction for ro-tor chains, even in the caseF= 0. Indeed, the
%    standardtechniques (see for instance [11, 12]) used to prove exis-tence
%    and uniqueness of an invariant measure for chainsof oscillators under
%    thermal forcing do not apply here
%
%
%    Some interesting relations can nonetheless be obtainedunder
%    the assumption that the stationary state exists.
%
%
%    % IfF= 0 andTL=TR=T, the system is inequi-librium, and the unique stationary measure is given bythe Gibbs measure
%
%\end{frame}


%
%        The Langevin dynamics may be seen as some modification of the Hamiltonian dynamics with
%        two added components: a damping term −γ(qt)M−1pt dt (dissipation) and a random forcing term
%        σ(qt)dWt (fluctuation). The energy dissipation due to damping is compensated by the random
%
%        stoltz page 33


%Of course, to reach some steady-state, some dissipation
%mechanism has to be considered as well, otherwise the external forcing may
%lead to an uncontrolled growth of the energy of the system.


%    Les propriétés thermodynamiques des états stationnaires hors équilibre
%    sont très peu comprises. Ces états sont généralement caractérisés par

%    des courants de quantités conservées (telles que l’énergie), qui circulent
%    dans le système. Lorsque les états stationnaires sont proches des états
%    d’équilibre, c’est-à-dire quand les perturbations sont de faible intensité,
%    la théorie de la réponse linéaire est efficace et explique les phénomènes
%    macroscopiques communs tels que la loi de Fourier. Ainsi, dans un système
%    en contact avec deux thermostats à températures différentes, si la différence
%    entre les deux températures est faible, le flux de chaleur est proportionnel
%    au gradient thermique. D’autre part, il n’y a pas de théorie générale pour
%    décrire les systèmes dans un état stationnaire loin de l’équilibre, et les
%    propriétés macroscopiques correspondantes semblent dépendre des détails
%    spécifiques de la dynamique.

%    A traditional way to implement the interaction with reservoirs amounts
%to introducing simul-taneously random forces and dissipation according
%to the generalprescription of fluctuation-dissipation theorem. This could
%be regarded as the limit case of the previous model whenγ±becomes very large.
%Consequently, the reservoirs are not affected by the system dynamics. In
%the simple case of an equal-mass chain, this results in the following set
%of Langevin equations

%    \gamma >0 determines the strength of the coupling to the thermostat
% (fluctuation-diffusion).

% In all schemes of heat baths there is at least one parameter controlling
%the coupling strength:let us generically call itg. It can either be the
%inverse of the average time between subsequentcollisions, or the
%dissipation rate λ in the Langevin equation

% https://arxiv.org/pdf/cond-mat/0112193.pdf page 15

% Ornstein-Uehlenbeck process


% Another prototypical feature of the Langevin equation is the occurrence of
%the damping coefficient λ {\displaystyle \lambda } \lambda in the correlation
%function of the random force, a fact also known as Einstein relation.


% For example, Albert Einstein noted in his 1905 paper on Brownian motion
%that the same random forces that cause the erratic motion of a particle
%in Brownian motion would also cause drag if the particle were pulled through
%the fluid. In other words, the fluctuation of the particle at rest has the
%same origin as the dissipative frictional force one must do work against,
%if one tries to perturb the system in a particular direction.

% fluctuation-dissipation


% hors equilibre - constats interressants
% calcul temperature (pdf utile - "energy continuity equation")
% loi fourier
% thermostat
% flux d'energie total
% forcage modelise quoi ???????????? c'est dans un de ces articles......


%    \bibliographystyle{plain}
%    \bibliography{sample}
%    \printbibliography
